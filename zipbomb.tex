\documentclass[letterpaper,twocolumn,10pt]{article}
\usepackage{usenix2019_v3}
\usepackage{graphicx}
\usepackage[binary-units,number-unit-product=~]{siunitx}
% \usepackage[hyphens]{url}
% \urlstyle{same}

% Disable metadata for reproducible PDF.
% https://tex.stackexchange.com/a/313605
\ifpdf
\pdfinfoomitdate=1
\pdftrailerid{}
\pdfsuppressptexinfo=-1
\hypersetup{pdfcreator={},pdfproducer={}}
\fi

\newcommand{\kB}{\kilo\byte}
\newcommand{\MB}{\mega\byte}
\newcommand{\GB}{\giga\byte}
\newcommand{\TB}{\tera\byte}
\newcommand{\PB}{\peta\byte}
\newcommand{\EB}{\exa\byte}
\newcommand{\EiB}{\exbi\byte}

\begin{document}

\date{}

\title{\Large \bf A better zip bomb}

\author{
% {\rm David Fifield}
}

\maketitle

\begin{abstract}
We show how to construct a
\emph{non-recursive} zip bomb
that achieves a high compression ratio by
overlapping files inside the zip container.
``Non-recursive'' means that it does not rely on
a decompressor's recursively unpacking zip files nested within zip files:
it expands fully after a single round of decompression.
The output size increases quadratically in the input size,
reaching a compression ratio of over 28~million
($\SI{10}{\MB} \rightarrow \SI{281}{\TB}$)
at the limits of the zip format.
Even greater expansion is possible using
64-bit extensions.
The construction uses only the most common compression algorithm, DEFLATE,
and is compatible with most zip implementations.
\end{abstract}


\section{Introduction}

Compression bombs that use the zip format
must cope with the fact that DEFLATE,
the only compression algorithm universally supported by zip parsers,
cannot achieve a compression ratio greater than
\num{1032}~\cite{zlib_tech}.
For this reason, zip bombs typically rely on recursive decompression,
nesting zip files within zip files to get an extra factor of 1032 with each layer.
But the trick only works on parsers that
unzip recursively, and most do not.
The best-known zip bomb, 42.zip~\cite{42.zip},
expands to a formidable \SI{4.5}{\PB}
if all six of its layers are recursively unzipped,
but a measly \SI{0.6}{\MB} at the top layer.
Zip quines, like those of
Ellingsen~\cite{ellingsen}
and Cox~\cite{cox},
contain a copy of themselves
and thus expand infinitely if recursively unzipped,
but are likewise perfectly safe to unzip once.

This article shows how to construct a non-recursive zip bomb
with a compression ratio that surpasses the DEFLATE limit of 1032.
It works by overlapping files inside the zip container,
in order to reference a ``kernel'' of highly compressed data multiple times,
without making multiple copies of it.
The output size grows quadratically in the input size; i.e.,
the compression ratio gets better as the input file gets bigger.
The construction depends on specific features of both zip and DEFLATE---it
is not portable to other file formats or compression algorithms.
It works with zip parsers that work in the standard way
of reading file locations from a central directory,
but not with ``streaming'' parsers that try to parse in one pass.

<p>
Our goals are:
</p>
<ul>
<li>
Maximize the compression ratio.
We define the compression ratio as the the sum of the sizes
of all the files contained the in the zip file,
divided by the size of the zip file itself.
We don't count filenames or other filesystem metadata,
only file contents.
</li>
<li>
Be compatible.
Zip is a tricky format and zip parsers differ especially
around edge cases and optional features.
Try to use only features that are universally supported.
We will look at some cases where efficiency can be increased
at the cost of compatibility.
</li>
</ul>

\section{Structure of a zip file}

\begin{figure*}
\includegraphics{figures/normal}
\caption{
The structure of a normal zip file.
}
\label{fig:normal}
\end{figure*}

\section{Overlapping files}

\cite{SAR-PR-2006-04}

\begin{figure*}
\includegraphics{figures/overlap}
\caption{
xxx
}
\label{fig:overlap}
\end{figure*}

\section{Quoting file headers}

\begin{figure*}
\includegraphics{figures/quote}
\caption{
xxx
}
\label{fig:quote}
\end{figure*}

% \section*{Acknowledgements}

\section*{Availability}

% \url{https://www.bamsoftware.com/hacks/zipbomb/}

\bibliographystyle{plain}
\bibliography{zipbomb}

\end{document}
