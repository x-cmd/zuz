\documentclass[letterpaper,twocolumn,10pt]{article}
\usepackage{usenix2019_v3}
\usepackage{graphicx}
\usepackage[binary-units,number-unit-product=~]{siunitx}
% \usepackage[hyphens]{url}
% \urlstyle{same}

% Disable metadata for reproducible PDF.
% https://tex.stackexchange.com/a/313605
\ifpdf
\pdfinfoomitdate=1
\pdftrailerid{}
\pdfsuppressptexinfo=-1
\hypersetup{pdfcreator={},pdfproducer={}}
\fi

\newcommand{\kB}{\kilo\byte}
\newcommand{\MB}{\mega\byte}
\newcommand{\GB}{\giga\byte}
\newcommand{\TB}{\tera\byte}
\newcommand{\PB}{\peta\byte}
\newcommand{\EB}{\exa\byte}
\newcommand{\EiB}{\exbi\byte}

\begin{document}

\date{}

\title{\Large \bf A better zip bomb}

\author{
% {\rm David Fifield}
}

\maketitle

\begin{abstract}
We show how to construct a
\emph{non-recursive} zip bomb
that achieves a high compression ratio by
overlapping files inside the zip container.
``Non-recursive'' means that it does not rely on
a decompressor's recursively unpacking zip files nested within zip files:
it expands fully after a single round of decompression.
The output size increases quadratically in the input size,
reaching a compression ratio of over 28~million
($\SI{10}{\MB} \rightarrow \SI{281}{\TB}$)
at the limits of what a zip file can represent.
Even greater performance is possible using
64-bit extensions to the zip format.
The construction uses only the most common compression algorithm, DEFLATE,
and is compatible with most zip implementations.
\end{abstract}


\section{Introduction}

A paragraph of text goes here. Lots of text. Plenty of interesting
text. Text text text text text text text text text text text text text
text text text text text text text text text text text text text text
text text text text text text text text text text text text text text
text text text text text text text.
More fascinating text. Features galore, plethora of promises.

\cite{190996}

\section{Structure of a zip file}

% \section*{Acknowledgements}

\section*{Availability}

% \url{https://www.bamsoftware.com/hacks/zipbomb/}

\bibliographystyle{plain}
\bibliography{zipbomb}

\end{document}
